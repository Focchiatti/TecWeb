\documentclass{tecweb}

\title{Relazione progetto Tecnologie Web}

\begin{document}
	\begin{titlepage}
		\begin{center}
			\includegraphics[width=5cm]{Logo2}
			\vspace{0.5cm}	\\
			\Huge \textbf{Serie-a-mente}
			\vspace{0.5cm}\\
			\normalsize \textbf{Progetto di Tecnologie Web}
			\vspace{0.7cm}	\\
			\renewcommand\arraystretch{1.3}	
			\begin{tabularx}{11cm}{r|X}
				\multicolumn{2}{c}{\textbf{Informazioni sul gruppo}}\\ 
				\hline
				\textbf{Componenti} & Marco Focchiatti \\ & Tommaso Loss \\ & Valentina Marcon \\  
				\textbf{Referente} & Valentina Marcon \\ 
				\textbf{Mail} & valentina.marcon3@studenti.unipd.it \\
			\end{tabularx}
			\vspace{0.7cm}
			\textbf{Indirizzo del sito:}
			\vspace{0.7cm}	\\
			\begin{tabularx}{11cm}{r|X}
				\multicolumn{2}{c}{\textbf{Dati Login}}\\ 
				\hline
				\textbf{Amministratore} & admin - admin \\
				\textbf{Utente} & user-user
			\end{tabularx}
		\vspace{2cm}
		\textit{Anno Accademico 2017-2018}
			
		\end{center}
	\end{titlepage}

	\section{Abstract}
	Lo scopo del progetto è la realizzazione di un sito che svolga la duplice funzione di contenitore di informazioni, generali e particolari, riguardo alle serie televisive e di gestire un basilare sistema di votazione da parte degli utenti. In questo modo si facilita la scelta da parte di un utente riguardo a una nuova serie da seguire e si garantisce la possibilità di rimanere aggiornati rispetto alle serie seguite. \\
	L'interazione è prevista solamente fra le classi di utenza e il sito a causa dell'ampio e variegato target al quale si fa riferimento. L'utente infatti può aggiungere o rimuovere serie alla sua collezione virtuale, e votarle. L'amministratore gestisce le serie presenti nel sito, e le notizie riportate. \\
	Proprio a causa della varietà di utenza prevista è stato data priorità all'usabilità e all'accessibilità, utilizzando XHTML strict, in modo da garantire l'utilizzo del sito a prescindere dal contesto, rispettando lo standard del W3C.
	\newpage
	\section{Utenti Destinatari}
	Il sito si rivolge al pubblico, variegato come età e cultura, delle serie televisive. Sia come utenti veri e proprio che come utilizzatori occasionali interessati a informazione più o meno approfondite, riguardo l'argomento. Solo gli utenti iscritti possono votare le serie ma non è necessario il login per vedere l'indice di gradimento di una serie. Considerando che i destinatari sono molto eterogenei abbiamo deciso di non consentire interazione fra utenti, sotto forma di commenti alle serie, in quanto necessiterebbe di un notevole lavoro di moderazione.
	\newpage
	
	
	\section{Accessibilità}
	\subsection{Separazione fra contenuto, presentazione e struttura}
	
	\newpage
	\section{Usabilità}
	\newpage
	\section{Gerarchia dei file}
	\newpage
	\section{Struttura}
	\newpage
	\section{Presentazione}
	\newpage
	\section{Gestione dei dati}
	\newpage
	\section{Validazione e Test}
	\newpage
	\appendix
	\section{Organizzazione del gruppo}
	
\end{document}